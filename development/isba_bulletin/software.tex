%!TEX TS-program = pdflatex
\documentclass{article}
\usepackage[round]{natbib}
\usepackage{graphicx}
\usepackage{pifont,fancybox,multicol,palatcm}
\usepackage{caption}
\usepackage{float}
\usepackage{fancyhdr}
\usepackage{geometry}
\pagestyle{fancy} 
\geometry{margin=1.2in,top=1in,bottom=1in}
\usepackage{amsmath,amssymb}
\usepackage{bm}
\usepackage{textcomp}
\usepackage[usenames,dvipsnames]{color}
\usepackage[pdfmenubar=false,pdftex,pdfpagelabels=true,colorlinks=true,linkcolor=ISBABlue,citecolor=ISBABlue,urlcolor=DarkOrange]{hyperref}
\usepackage[export]{adjustbox}
\usepackage{soul}
\usepackage{wrapfig}
\usepackage{subcaption}
\usepackage{tabularx, booktabs}
%%%% Features this issue
\usepackage{pgfplots}
\usepackage{tikz}
\usepackage{url}
\usepackage{placeins}
%%%%




\graphicspath{{figures/}}

\usepackage{framed}
\makeatletter
\newenvironment{kframe}{%
 \def\at@end@of@kframe{}%
 \ifinner\ifhmode%
  \def\at@end@of@kframe{\end{minipage}}%
  \begin{minipage}{\columnwidth}%
 \fi\fi%
 \def\FrameCommand##1{\hskip\@totalleftmargin \hskip-\fboxsep
 \colorbox{shadecolor}{##1}\hskip-\fboxsep
     % There is no \\@totalrightmargin, so:
     \hskip-\linewidth \hskip-\@totalleftmargin \hskip\columnwidth}%
 \MakeFramed {\advance\hsize-\width
   \@totalleftmargin\z@ \linewidth\hsize
   \@setminipage}}%
 {\par\unskip\endMakeFramed%
 \at@end@of@kframe}
\makeatother

\parskip 1.5ex
\parindent 0em

\newcommand{\para}[1]{\vspace{.3em}\textbf{#1}\par}

\newcommand{\issue}{ISBA Bulletin, \textbf{32}(3), September 2025}
\newcommand{\blue}[1]{\textcolor{blue}{#1}}

\newcommand{\MyFilledBox}[1]{
\label{#1}
\vspace{1em}
\markright{\emph{#1}}{}
\hypertarget{#1}{}
\fcolorbox{White}{LightBlue}
{\begin{minipage}[t]{15.3cm}
\begin{center}

	\vspace{2.4mm}
	\subsection*{#1}

	\vspace{.3mm}
\end{center}
\end{minipage}}
\vspace{1mm}}

\newcommand{\MyFilledbox}[2]{
\hspace{-2.3mm}
\fcolorbox{White}{LightBlue}
{
\begin{minipage}[t]{#2}
	\begin{center}
	
		\textbf{\large #1}
	
	\end{center}
\end{minipage}}\\[2mm]}

\newcommand{\MyFancyBox}[3]{
\hspace{-4mm}
\fcolorbox{Black}{VeryLightBlue}{
\begin{minipage}[t]{#3}
	\MyFilledbox{#1}{#3}\\
	#2
\end{minipage}
}}

\newcommand{\MyFancyBoxWhite}[3]{
\hspace{-4mm}
\fcolorbox{Black}{White}{
\begin{minipage}[t]{#3}
	\MyFilledbox{#1}{#3}\\
	#2
\end{minipage}
}}

\renewcommand{\sectionmark}[1]{\markboth{#1}{}}

\definecolor{ISBABlue}{rgb}{0, .122, .388}
\definecolor{LightBlue}{rgb}{0.70, 0.75, 0.85}
\definecolor{VeryLightBlue}{rgb}{0.95,0.95,0.99}
\definecolor{DarkOrange}{rgb}{.9, 0.39, 0.0}

\fancyhead[L]{\issue}
\fancyhead[R]{}
\fancyfoot[C]{\thepage}
\fancyfoot[R]{\tt \href{https://www.bayesian.org}{www.bayesian.org}}

%---------- packages and commands for code blocks @software
\usepackage{mathtools}
\usepackage{fancyvrb}
\newcommand{\VerbBar}{|}
\newcommand{\VERB}{\Verb[commandchars=\\\{\}]}
\DefineVerbatimEnvironment{Highlighting}{Verbatim}{commandchars=\\\{\}}
% Add ',fontsize=\small' for more characters per line
\definecolor{shadecolor}{RGB}{248,248,248}
\newenvironment{Shaded}{\begin{snugshade}}{\end{snugshade}}
\newcommand{\AlertTok}[1]{\textcolor[rgb]{0.94,0.16,0.16}{#1}}
\newcommand{\AnnotationTok}[1]{\textcolor[rgb]{0.56,0.35,0.01}{\textbf{\textit{#1}}}}
\newcommand{\AttributeTok}[1]{\textcolor[rgb]{0.13,0.29,0.53}{#1}}
\newcommand{\BaseNTok}[1]{\textcolor[rgb]{0.00,0.00,0.81}{#1}}
\newcommand{\BuiltInTok}[1]{#1}
\newcommand{\CharTok}[1]{\textcolor[rgb]{0.31,0.60,0.02}{#1}}
\newcommand{\CommentTok}[1]{\textcolor[rgb]{0.56,0.35,0.01}{\textit{#1}}}
\newcommand{\CommentVarTok}[1]{\textcolor[rgb]{0.56,0.35,0.01}{\textbf{\textit{#1}}}}
\newcommand{\ConstantTok}[1]{\textcolor[rgb]{0.56,0.35,0.01}{#1}}
\newcommand{\ControlFlowTok}[1]{\textcolor[rgb]{0.13,0.29,0.53}{\textbf{#1}}}
\newcommand{\DataTypeTok}[1]{\textcolor[rgb]{0.13,0.29,0.53}{#1}}
\newcommand{\DecValTok}[1]{\textcolor[rgb]{0.00,0.00,0.81}{#1}}
\newcommand{\DocumentationTok}[1]{\textcolor[rgb]{0.56,0.35,0.01}{\textbf{\textit{#1}}}}
\newcommand{\ErrorTok}[1]{\textcolor[rgb]{0.64,0.00,0.00}{\textbf{#1}}}
\newcommand{\ExtensionTok}[1]{#1}
\newcommand{\FloatTok}[1]{\textcolor[rgb]{0.00,0.00,0.81}{#1}}
\newcommand{\FunctionTok}[1]{\textcolor[rgb]{0.13,0.29,0.53}{\textbf{#1}}}
\newcommand{\ImportTok}[1]{#1}
\newcommand{\InformationTok}[1]{\textcolor[rgb]{0.56,0.35,0.01}{\textbf{\textit{#1}}}}
\newcommand{\KeywordTok}[1]{\textcolor[rgb]{0.13,0.29,0.53}{\textbf{#1}}}
\newcommand{\NormalTok}[1]{#1}
\newcommand{\OperatorTok}[1]{\textcolor[rgb]{0.81,0.36,0.00}{\textbf{#1}}}
\newcommand{\OtherTok}[1]{\textcolor[rgb]{0.56,0.35,0.01}{#1}}
\newcommand{\PreprocessorTok}[1]{\textcolor[rgb]{0.56,0.35,0.01}{\textit{#1}}}
\newcommand{\RegionMarkerTok}[1]{#1}
\newcommand{\SpecialCharTok}[1]{\textcolor[rgb]{0.81,0.36,0.00}{\textbf{#1}}}
\newcommand{\SpecialStringTok}[1]{\textcolor[rgb]{0.31,0.60,0.02}{#1}}
\newcommand{\StringTok}[1]{\textcolor[rgb]{0.31,0.60,0.02}{#1}}
\newcommand{\VariableTok}[1]{\textcolor[rgb]{0.00,0.00,0.00}{#1}}
\newcommand{\VerbatimStringTok}[1]{\textcolor[rgb]{0.31,0.60,0.02}{#1}}
\newcommand{\WarningTok}[1]{\textcolor[rgb]{0.56,0.35,0.01}{\textbf{\textit{#1}}}}
%----------

\begin{document}
\hypertarget{TOP}{}
\thispagestyle{empty}
{\hfill \sc September 2025}

\vspace*{-.7em}
{\hfill \sc Vol.\ 32 No.\ 3}

\vspace*{-1.5em}
\begin{center}
\includegraphics[width=7cm]{isbalogo}
\end{center}
\fcolorbox{Black}{LightBlue}
{
\begin{minipage}[t]{15.1cm}
	\begin{center}	  
		\vspace{.5em}
%		\includegraphics[width=5cm]{isbalogotrans}\\[2em]
		\textbf{\Huge{ \sc{The \hspace{.1mm} ISBA\hspace{3mm} Bulletin}}}\\[2mm]
%		 		 {\sc Vol.\ 28 No.\ 1 \hfill  March 2021}\\[1pc] 				 
		 		 {\sc {\large Official bulletin of the International Society for Bayesian Analysis}}
		 		 \vspace{.5em}
	\end{center}
\end{minipage}
}


\clearpage

%%%%%%%%%%%%%%%%%%%%%%%%%%%%%%%%%%%%%%%%%%%%%%%

\MyFilledBox{SOFTWARE HIGHLIGHT}
\begin{center}
{\large Luca\enspace Porta Mana} \\
{\tt \href{mailto:pgl@portamana.org}{pgl@portamana.org}}

\bigskip

\Large
\textsc{inferno\\
INFERence in R with Bayesian NOnparametrics}
\end{center}
\newcommand{\citebi}{\cite}
\newcommand{\citein}{\cite}
\newcommand*{\subtitleproc}[1]{}
\newcommand*{\chapb}{ch.}

Many researchers in important fields, such as medicine, sadly still use $p$-values and frequentist statistics to do \emph{population inference}, despite their intrinsic flaws. Some researchers use Bayesian methods but limit themselves to parametric ones, which make possibly unrealistic statistical assumptions.

Until a couple decades ago such practices could somehow be justified by pragmatic reasons:
\begin{itemize}
\item Better methods were computationally too expensive.
\item The population-inference problems were low-dimensional; one could \emph{visually} check whether the method or assumptions were appropriate to the problem, and change them otherwise, or consider the results as simply qualitative.
\end{itemize}

But today, in many cases, the reasons above cannot meaningfully be given anymore:
\begin{itemize}
\item Bayesian methods, even nonparametric, have become computationally feasible for many inference problems.
\item Many population-inference problems today involve tens or hundreds of variates of different kinds, and are therefore very high-dimensional. It is impossible to visually check whether frequentist results or parametric assumptions are acceptable, or by how much they err. Results may therefore be affected by large errors \citep{draper1995}, whose existence is often not even reported.
\end{itemize}

There is one reason that can still be given today for not using Bayesian methods, especially parametric ones, in population inference: \emph{lack of user-friendly software}. Clinicians who'd be curious to try out Bayesian analysis of their studies simply can't, because that would require the study of Markov-chain Monte Carlo techniques, of programming languages to implement them, and of a plethora of debated methods and acquired visual skills to ``assess convergence''. Most clinicians don't have time to learn all this even if they wanted to.

Available packages for Bayesian nonparametrics focus moreover on the problem of \emph{regression}, that is, the inference of a functional relationship -- assumed to exist -- between a set of predictor variates and a target or ``predictand'' variate. These packages are not appropriate to \emph{population inference}, where no functional relationships exist a priori, and where often there is no a priori division between predictor and predictand variates. In clinical studies one may be interested in calculating the probability of an effect or symptom given a condition, and of a condition given an effect or symptom.

The R-package \textbf{inferno} was built to try to fill the need for this kind of software.







\citep{walker2010}

\bibliographystyle{abbrvnat}
\bibliography{../manual/portamanabib.bib}

\end{document}
